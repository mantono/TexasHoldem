\avsnitt{Ekvivalensklassuppdelning -- subtractChips}

En kort presentation av vad ni valt ut f�r att till�mpa ekvivalensklassuppdelning p�. Ni ska kort motivera valet, och ge tillr�ckligt med information f�r att det ska g� att bed�ma er. Detta avsnitt och de tre f�jande (till och med testmatrisen) ska finnas f�r samtliga delar ni till�mpat ekvivalensklassuppdelning p�.

\avsnitt{Ekvivalensklasser -- subtractChips}

Samtliga ekvivalensklasser f�r denna del presenterade p� ett tydligt s�tt.\\ public boolean subtractChips(int bet)

\begin{table}[h]
\begin{tabular}{|l|l|l|l|}
\hline
\textbf{Invalida interval} \cellcolor{red!15}	& \cellcolor{red!15}			& \textbf{Valida interval} \cellcolor{green!15}	& \cellcolor{green!15}					\\ \hline
int bet \cellcolor{red!15}						& $<=1$ \cellcolor{red!15}		& int bet \cellcolor{green!15}					& $0<bet<=chips$ \cellcolor{green!15}	\\ \hline
int bet	\cellcolor{red!15}						& $>chips$ \cellcolor{red!15}	& \cellcolor{green!15}							& \cellcolor{green!15}					\\ \hline
\end{tabular}
\end{table}



\avsnitt{Testfall -- subtractChips}

Testfallen som ni f�tt fram fr�n ekvivalensklasserna. Observera att vi inte vill ha n�gon kod h�r, utan bara en tydlig presentation av testfallen i n�gon l�mplig tabellform.
\begin{table}[h]
\begin{tabular}{llll}
\textbf{Valid test cases} &                  &  &                 \\
id                        & \textbf{Inputs}  &  & \textbf{Result} \\
                          & \textit{int bet} &  &                 \\
1                         & 1                &  & TRUE            \\
2                         & chips            &  & TRUE            \\
                          &                  &  &                
\end{tabular}
\end{table}
\\
\begin{table}[h]
\begin{tabular}{lllll}
\textbf{Invalid test cases} &                 &  &  &                 \\
id                          & \textbf{Inputs} &  &  & \textbf{Result} \\
                            & int bet         &  &  &                 \\
1                           & 0               &  &  & FALSE           \\
2                           & chips + 1       &  &  & FALSE          
\end{tabular}
\end{table}

\avsnitt{Testmatris -- subtractChips}

En testmatris som visar sambandet mellan ekvivalensklasserna och testfallen f�r denna del.

\avsnitt{Ekvivalensklassuppdelning - Game-konstruktor}

En kort presentation av vad ni valt ut f�r att till�mpa ekvivalensklassuppdelning p�. Ni ska kort motivera valet, och ge tillr�ckligt med information f�r att det ska g� att bed�ma er. Detta avsnitt och de tre f�jande (till och med testmatrisen) ska finnas f�r samtliga delar ni till�mpat ekvivalensklassuppdelning p�.

\avsnitt{Ekvivalensklasser -- Game-konstruktor }

Samtliga ekvivalensklasser f�r denna del presenterade p� ett tydligt s�tt.

\begin{table}[h]
\begin{tabular}{|l|l|l|l|}
\hline
\cellcolor{red!15} \textbf{Invalida interval}	& \cellcolor{red!15}					& \cellcolor{green!15} \textbf{Valida interval}		& \cellcolor{green!15}		\\ \hline
\cellcolor{red!15} int bigblind					& \cellcolor{red!15} $<=0$				& \cellcolor{green!15} int bigBlind					& \cellcolor{green!15} $>0$	\\ \hline
\cellcolor{red!15} double blindsRaisePercentage & \cellcolor{red!15} $<0$				& \cellcolor{green!15} double blindsRaisePercentage	& \cellcolor{green!15} $>=0$\\ \hline
\cellcolor{red!15} Players... players			& \cellcolor{red!15} null				& \cellcolor{green!15} Players... players			& \cellcolor{green!15}*		\\ \hline
\end{tabular}
\end{table}

\avsnitt{Testfall -- Game-konstruktor }

Testfallen som ni f�tt fram fr�n ekvivalensklasserna. Observera att vi inte vill ha n�gon kod h�r, utan bara en tydlig presentation av testfallen i n�gon l�mplig tabellform.
?
\avsnitt{Testmatris -- Game-konstruktor }

En testmatris som visar sambandet mellan ekvivalensklasserna och testfallen f�r denna del.

