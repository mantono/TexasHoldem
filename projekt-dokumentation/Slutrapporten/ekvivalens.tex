\avsnitt{Ekvivalensklassuppdelning -- subtractChips}

En kort presentation av vad ni valt ut f�r att till�mpa ekvivalensklassuppdelning p�. Ni ska kort motivera valet, och ge tillr�ckligt med information f�r att det ska g� att bed�ma er. Detta avsnitt och de tre f�jande (till och med testmatrisen) ska finnas f�r samtliga delar ni till�mpat ekvivalensklassuppdelning p�. 

\avsnitt{Ekvivalensklasser -- subtractChips}

Samtliga ekvivalensklasser f�r denna del presenterade p� ett tydligt s�tt.\\ public boolean subtractChips(int bet)

\begin{table}[h]
\begin{tabular}{llll}
\textbf{Invalid intervals} &                   & \textbf{Valid intervals} &                              \\
int bet                    & \textless=1       & int bet                  & 0\textlessbet\textless=chips \\
int bet                    & \textgreaterchips &                          &                              \\
                           &                   &                          &                              \\
                           &                   &                          &                              \\
                           &                   &                          &                             
\end{tabular}
\end{table} 



\avsnitt{Testfall -- subtractChips}

Testfallen som ni f�tt fram fr�n ekvivalensklasserna. Observera att vi inte vill ha n�gon kod h�r, utan bara en tydlig presentation av testfallen i n�gon l�mplig tabellform. 
?
\avsnitt{Testmatris -- subtractChips}

En testmatris som visar sambandet mellan ekvivalensklasserna och testfallen f�r denna del.

\avsnitt{Ekvivalensklassuppdelning � Game-konstruktor}

En kort presentation av vad ni valt ut f�r att till�mpa ekvivalensklassuppdelning p�. Ni ska kort motivera valet, och ge tillr�ckligt med information f�r att det ska g� att bed�ma er. Detta avsnitt och de tre f�jande (till och med testmatrisen) ska finnas f�r samtliga delar ni till�mpat ekvivalensklassuppdelning p�. 

\avsnitt{Ekvivalensklasser -- Game-konstruktor }

Samtliga ekvivalensklasser f�r denna del presenterade p� ett tydligt s�tt.

\avsnitt{Testfall -- Game-konstruktor }

Testfallen som ni f�tt fram fr�n ekvivalensklasserna. Observera att vi inte vill ha n�gon kod h�r, utan bara en tydlig presentation av testfallen i n�gon l�mplig tabellform. 
?
\avsnitt{Testmatris -- Game-konstruktor }

En testmatris som visar sambandet mellan ekvivalensklasserna och testfallen f�r denna del.

