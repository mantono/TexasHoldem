\avsnitt{Introduktion}


Vi valde att inte skriva ett av de projekt som vi blev givna utan b�rjade skriva ett klassbibliotek
till ett kortspel, eller mer specifikt till texashold'em poker. Vi anv�nde oss utav Git som
versionshanteringsverktyg under hela projektets g�ng. Ungef�r halvv�gs in i projektet gjordes en
formell granskning av majoriteten av koden som vi hade skrivit f�r att senare utveckla det vi redan
hade skrivit till en b�ttre slutprodukt. F�r att kolla p� statistik i v�r kod anv�nde vi Eclipse Metrics Plugin
som ett statiskt m�tt. Som dynamiskt m�tt f�r att uppm�ta prestanda som ett antal olika metoder drar
anv�nde vi oss utav JVM Monitor i Eclipse.
F�r att uppn� h�gst m�jliga coverage anv�nde vi oss utav EclEmma f�r att kolla att vi uppn�dde 100\% coverage p� v�ra klasser.
Under kodkritik k�rde vi igenom v�ra klasser med FindBugs f�r att se vad vi inte redan hade hittat
som kunde �ndras och m�jligen g�ras b�ttre.

K�llkod: \url{http://81.225.210.252/texas.tar.gz}

\begin{itemize}
	\item EclEmma - coverage
	\item Eclipse Metrics Plugin - software metrics
	\item GIT - versionshanteringssystem
	\item JUnit - enhetstestning
	\item FindBugs - kodkritik
	\item JVM Monitor - profiler
	\item Jflap - tillst�ndsmaskin
	\item ANT - byggscript
	\item Eclipse inbyggda debugger - debugger
\end{itemize}
