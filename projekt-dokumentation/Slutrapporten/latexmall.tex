\documentclass[a5paper,10pt,oneside,landscape]{article}

\usepackage[swedish]{babel}
\usepackage[T1]{fontenc}
\usepackage[latin1]{inputenc} % Denna m�ste �ndras till r�tt teckenkodning f�r er

\usepackage{graphicx}
\usepackage{url}
%\usepackage{ifthen}
\usepackage[a5paper,margin=1.5cm,noheadfoot]{geometry}

\usepackage{listings}
\lstset{language=Java,numbers=left,frame=L,floatplacement=hbtp}
\def \lstlistingname {Kodexempel}

\usepackage{ifpdf}
\ifpdf
	\usepackage[hidelinks]{hyperref}
\else
	\usepackage{url}
\fi
\newcommand{\avsnitt}[1]{\newpage\section*{#1}}

\begin{document}

%%%%%%%%%%%%% Framsidan %%%%%%%%%%%%%%

\title{Projektnamn}
\author{Adam Alfa\\adal1234 \and Beatrice Beta\\bebe5678 \and Caesar Gamma\\caga9012 \and Diana Delta\\dide3456}
\maketitle

%%%%%%%%%%%%% Rapportens delar %%%%%%%%%%%%%%%

\avsnitt{Introduktion}


Vi valde att inte skriva ett av de projekt som vi blev givna utan b�rjade skriva ett klassbibliotek
till ett kortspel, eller mer specifikt till texashold'em poker. Vi anv�nde oss utav Git som
versionshanteringsverktyg under hela projektets g�ng. Ungef�r halvv�gs in i projektet gjordes en
formell granskning av majoriteten av koden som vi hade skrivit f�r att senare utveckla det vi redan
hade skrivit till en b�ttre slutprodukt. F�r att kolla p� statistik i v�r kod anv�nde vi Eclipse Metrics Plugin
som ett statiskt m�tt. Som dynamiskt m�tt f�r att uppm�ta prestanda som ett antal olika metoder drar
anv�nde vi oss utav JVM Monitor i Eclipse.
F�r att uppn� h�gst m�jliga coverage anv�nde vi oss utav EclEmma f�r att kolla att vi uppn�dde 100\% coverage p� v�ra klasser.
Under kodkritik k�rde vi igenom v�ra klasser med FindBugs f�r att se vad vi inte redan hade hittat
som kunde �ndras och m�jligen g�ras b�ttre.

K�llkod: \url{http://81.225.210.252/texas.tar.gz}

\begin{itemize}
	\item EclEmma - coverage
	\item Eclipse Metrics Plugin - software metrics
	\item GIT - versionshanteringssystem
	\item JUnit - enhetstestning
	\item FindBugs - kodkritik
	\item JVM Monitor - profiler
	\item Jflap - tillst�ndsmaskin
	\item ANT - byggscript
	\item Eclipse inbyggda debugger - debugger
\end{itemize}

\avsnitt{Slutlig design}
Vi valde att dela upp v�rat projekt p� ett s�dant sett att det finns ett separat paket som heter \emph{cards} vilket �r t�nkt som ett allm�nt kortspels-paket med kort, kortlekar, spelare, h�nder osv och sedan ett paket, \emph{texasholdem}, som �r mer specifikt inriktat p� kortspelet Texas Holdem. I det senare ligger ocks� s�dant som mer styr spelregler och spelmekaniken.

Nedan �r ett klassdiagram �ver hela v�rt projekt:\\
\DeclareGraphicsExtensions{.png}
\includegraphics[natwidth=848, natheight=958, scale=0.73, angle=0, trim = 0mm 0mm 42mm 16mm]{bilder/diagram/klassdiagram-2014-10-28}

Samt n�gra mer detaljerade diagram �ver projektet;

\DeclareGraphicsExtensions{.png}
\includegraphics[natwidth=623, natheight=638, scale=0.91, angle=0, trim = 0mm 0mm 0mm 26mm]{bilder/diagram/card1}

\DeclareGraphicsExtensions{.png}
\includegraphics[natwidth=590, natheight=532, scale=0.95, angle=0, trim = 0mm 0mm 0mm 5mm]{bilder/diagram/card2}

\DeclareGraphicsExtensions{.png}
\includegraphics[natwidth=640, natheight=678, scale=0.84, angle=0, trim = 0mm 0mm 0mm 19mm]{bilder/diagram/texas1}

\DeclareGraphicsExtensions{.png}
\includegraphics[natwidth=372, natheight=422, scale=1, angle=0, trim = 0mm 0mm 0mm 0mm]{bilder/diagram/texas2}

\avsnitt{Testdriven utveckling -- process}

En �versikt �ver hur ni till�mpat TDD med exempel fr�n olika personer och olika faser i projektet. Om ni har anv�nt versionshanteringssystemet ordentligt b�r all information som efterfr�gas h�r finnas i det. T�nk p� att kodexemplen ska vara l�sbara. 



\avsnitt{Testdriven utveckling -- erfarenheter}

En diskussion om vilka era erfarenheter ni dragit av att till�mpa TDD. Det finns inget r�tt eller fel h�r. Enda s�ttet att bli underk�nda �r att bara fuska �ver punkten och s�ga n�got pliktskyldigt.

Vi har arbetat med testdriven utveckling n�stan helt och h�llet genom projektet vilket har gett oss en m�ngd erfarenheter av hur bra/d�ligt det kan fungera. Till en b�rjan k�ndes det v�ldigt �verfl�digt och framtvingat d� man arbetade med simpla metoder, klasser och konstruktioner som k�ndes sj�lvklara hur man skulle skriva och som alltid fungerade direkt. Desto mer vi arbetade och kom in i sv�rare metoder och konstruktioner s� b�rjade testdriven utveckling k�nnas mer och mer anv�ndbart och mot slutet i projektet, speciellt n�r vi arbetade med klasses TexasRules, s� k�ndes det n�stan som att det hade varit om�jligt att genomf�ra den delen utan testdrivenutveckling och testfall. Eftersom man i klassen TexasRules beh�vde testa v�ldigt specifika fall, som t ex att r�tt person vinner om b�da har stege men en av personerna har h�gre kort, s� gjorde testfall att man enkelt kunde s�tta upp ett eller flera komplexa fall som endast g�tt igenom om koden �r korrekt skriven. Detta skulle vara v�ldigt sv�rt att testa utan testfall.

Testdriven utveckling ledde ocks� till att vi snabbt uppt�ckte fel som introducerats i koden. Detta eftersom man kunde se om ett testfall g�tt igenom tidigare antingen i commit-kommentarerna eller f�r att man sj�lv k�rt dem tidigare och d� ocks� se om man kan ha introducerat n�got fel i en specifik del av koden. Lyckades man d� inte se vilket fel man introducerat s� gick det alltid att titta p� versionen som commitats t ex med kommentaren �testfall X g�r nu igenom� och se hur koden s�g ut i det tillf�llet.

En ytterliggare positiv effekt av testdriven utveckling som vi inte t�nkt p� i f�rhand �r att den g�r koden mycket mer l�ttf�rst�elig f�r de man arbetar med d� man kan se hur koden �r t�nkt att fungera �ven om den inte i nuvarande l�ge fungerar. Detta leder till att man enkelt kan ta �ver metoder och skriva klart dem �ven om det inte var man sj�lv som b�rjade skriva p� den. �ven om man har dokumentation om ungef�r vad metoden ska g�ra s� ger testfallen mycket med information om exakt hur metoden ska bete sig och vad som ska komma ut ur den vid specifika l�gen.

\avsnitt{Ekvivalensklassuppdelning -- subtractChips}

En kort presentation av vad ni valt ut f�r att till�mpa ekvivalensklassuppdelning p�. Ni ska kort motivera valet, och ge tillr�ckligt med information f�r att det ska g� att bed�ma er. Detta avsnitt och de tre f�jande (till och med testmatrisen) ska finnas f�r samtliga delar ni till�mpat ekvivalensklassuppdelning p�.

\avsnitt{Ekvivalensklasser -- subtractChips}

Samtliga ekvivalensklasser f�r denna del presenterade p� ett tydligt s�tt.\\ public boolean subtractChips(int bet)

\begin{table}[h]
\begin{tabular}{|l|l|l|l|}
\hline
\textbf{Invalida interval} \cellcolor{red!15}	& \cellcolor{red!15}			& \textbf{Valida interval} \cellcolor{green!15}	& \cellcolor{green!15}					\\ \hline
int bet \cellcolor{red!15}						& $<=1$ \cellcolor{red!15}		& int bet \cellcolor{green!15}					& $0<bet<=chips$ \cellcolor{green!15}	\\ \hline
int bet	\cellcolor{red!15}						& $>chips$ \cellcolor{red!15}	& \cellcolor{green!15}							& \cellcolor{green!15}					\\ \hline
\end{tabular}
\end{table}



\avsnitt{Testfall -- subtractChips}

Testfallen som ni f�tt fram fr�n ekvivalensklasserna. Observera att vi inte vill ha n�gon kod h�r, utan bara en tydlig presentation av testfallen i n�gon l�mplig tabellform.
?
\avsnitt{Testmatris -- subtractChips}

En testmatris som visar sambandet mellan ekvivalensklasserna och testfallen f�r denna del.

\avsnitt{Ekvivalensklassuppdelning - Game-konstruktor}

En kort presentation av vad ni valt ut f�r att till�mpa ekvivalensklassuppdelning p�. Ni ska kort motivera valet, och ge tillr�ckligt med information f�r att det ska g� att bed�ma er. Detta avsnitt och de tre f�jande (till och med testmatrisen) ska finnas f�r samtliga delar ni till�mpat ekvivalensklassuppdelning p�.

\avsnitt{Ekvivalensklasser -- Game-konstruktor }

Samtliga ekvivalensklasser f�r denna del presenterade p� ett tydligt s�tt.

\begin{table}[h]
\begin{tabular}{|l|l|l|l|}
\hline
\cellcolor{red!15} \textbf{Invalida interval}	& \cellcolor{red!15}					& \cellcolor{green!15} \textbf{Valida interval}		& \cellcolor{green!15}		\\ \hline
\cellcolor{red!15} int bigblind					& \cellcolor{red!15} $<=0$				& \cellcolor{green!15} int bigBlind					& \cellcolor{green!15} $>0$	\\ \hline
\cellcolor{red!15} double blindsRaisePercentage & \cellcolor{red!15} $<0$				& \cellcolor{green!15} double blindsRaisePercentage	& \cellcolor{green!15} $>=0$\\ \hline
\cellcolor{red!15} Players... players			& \cellcolor{red!15} null				& \cellcolor{green!15} Players... players			& \cellcolor{green!15}*		\\ \hline
\end{tabular}
\end{table}

\avsnitt{Testfall -- Game-konstruktor }

Testfallen som ni f�tt fram fr�n ekvivalensklasserna. Observera att vi inte vill ha n�gon kod h�r, utan bara en tydlig presentation av testfallen i n�gon l�mplig tabellform.
?
\avsnitt{Testmatris -- Game-konstruktor }

En testmatris som visar sambandet mellan ekvivalensklasserna och testfallen f�r denna del.


\avsnitt{Tillst�ndsbaserad testning}
N�r vi disskuterade vilken del av programmet att anv�nda f�r den tillst�ndsbaserade testningen s� stod valet mellan boolean-attributet \emph{inRound} i klassen Player, eller hela Game-klassen. Tillslut valde vi attributet \emph{inRound}, med motivationen att \emph{inRound}s tillst�nd var tillr�ckligt viktiga f�r att skapa m�nga testfall, men samtidigt tillr�ckligt avgr�nsande s� att denna del inte blev f�r stort. Game-klassen hade helt enkelt f�r m�nga kombinationer av tilst�nd f�r att vara passande.

N�r vi ritade upp fl�desdiagramet f�r \emph{inRound} s� m�rkte vi att \emph{inRound} var beroende p� om Player hade n�gra chips eller inte. Dvs om Player inte hade chips, s� agerar klassen Player annorlunda i b�da \emph{inRound} tillst�nden, och d�rf�r m�ste vi inkludera detta p� n�got s�tt. Vi b�rjade fundera p� om vi skulle ha tv� tillst�ndsmaskiner gentemot varandra, med \emph{broar} mellan de olika tillst�nderna, men vi kom fram till att den implementationen var �verfl�dig, d� om Player f�r slut p� chips utan att vara \emph{inRound} s� borde den Playern inte finnas kvar i programmet �verhuvudtaget. S� vi gjorde ist�llet ytterligare ett tillst�nd som vi kallade \emph{Zero chips} f�r att vissa att Player befinner sig i den enda relevanta kombinationen av tillst�nd: den n�r Player �r \emph{inGame}, men har inga chips kvar (dvs g�tt all-in).

N�r vi var klara med diagramet disskuterade vi vilket t�ckningskriteruim vi skulle anv�nda oss utav. Valet var denna g�ng mellan state coverage eller path coverage. I efterhand ins�g vi hur dum denna disskussion egentligen var, d� path coverage automatiskt uppn�r state coverage, s� det blev sj�lvklart path coverage. Tankeg�ngarna var att det faktiskt inte var tillst�nden i sig som var det viktiga att testa, utan �verg�ngarna mellan de olika tillst�nden var de som var viktiga.

Modellen som vi designade visar p� hur klassen hanterar en spelare i v�rt system och vilka tillst�nd denne spelare kan befinna sig i.


\DeclareGraphicsExtensions{.png}
\includegraphics[natwidth=12559, natheight=592, scale=0.6, angle=0, trim = 35mm 2mm 0mm 0mm, clip]{bilder/diagram/state_machine}\\

\avsnitt{Testfall f�r tillst�ndsbaserad testning}

Testfallen som ni f�tt fram fr�n tillst�ndsmaskinen. Observera att vi inte vill ha n�gon kod h�r,
utan bara en tydlig presentation av testfallen i n�gon l�mplig tabellform.
Det ska enkelt g� att mappa testfallen till tillst�ndsmaskinen.

Eftersom att vi anv�nde testdriven utveckling n�r vi skrev Player och Game klasserna s� hade vi redan i princip testfall f�r boolean \emph{inGame}, och d�rf�r s� skapade vi en egen test-klass som vi enbart anv�nde som en kompletering.
Vi var lite os�kra p� hur m�nga testfall vi skulle g�ra. Skulle vi ha f� testfall, d�r varje testfall t�cker flera \emph{v�gar}, eller skulle vi ha ett testfall f�r varje individuell \emph{v�g}?
Vi besl�t oss f�r att g�ra m�nga testfall som testade en individuell sak, d� om det blir fel p� n�gon av de mittersta v�garna s� kommer f� fast l�nga testfall att inte testa de v�garna efter den misslyckade v�gen.
P� det h�r s�ttet �r det garanterat att vi t�cker alla v�gar, oberoende p� vilka v�gar som misslyckas eller inte.


testfallen i n�gon l�mplig tabellform

\begin{table}[h]
\begin{tabular}{|l|l|l|l|}
\hline
\textbf{I tillst�nd}	& \textbf{B�ge}		& \textbf{Till tillst�nd}	& \textbf{Testfall} \\ \hline
q0: Not in round		& Enter round		& q1: In round				& notInRoundToInRound\_EnterRoundTest()	\\ \hline
q0: Not in round		& Check			& q2: Illegal state			& notInRoundToIllegalState\_CheckTest()		\\ \hline
q0: Not in round		& All in			& q2: Illegal state			& notInRoundToIllegalState\_AllInTest()		\\ \hline
q0: Not in round		& Raise			& q2: Illegal state			& notInRoundToIllegalState\_RaiseTest()		\\ \hline
q0: Not in round		& Call			& q2: Illegal state			& notInRoundToIllegalState\_CallTest()		\\ \hline
q0: Not in round		& Fold			& q2: Illegal state			& notInRoundToIllegalState\_FoldTest()		\\ \hline
q1: In round			& Fold			& q0: Not in round			& inRoundToNotInRound_FoldTest()			\\ \hline
q1: In round			& Round over		& q0: Not in round			& inRoundToNotInRound_RoundOverTest()		\\ \hline
q1: In round			& Call			& q1: In round				& inRoundToInRound_CallTest()				\\ \hline
q1: In round			& Check			& q1: In round				& inRoundToInRound_CheckTest()			\\ \hline
q1: In round			& Raise			& q1: In round				& inRoundToInRound_RaiseTest()			\\ \hline
q1: In round			& All in			& q3: Zero chips			& inRoundToZeroChips_AllInTest()			\\ \hline
q3: Zero chips			& Call			& q2: Illegal state			& zeroChipsToIllegalState_call()				\\ \hline
q3: Zero chips			& All in			& q2: Illegal state			& zeroChipsToIllegalState_allIn()			\\ \hline
q3: Zero chips			& Fold			& q2: Illegal state			& zeroChipsToIllegalState_fold()				\\ \hline
q3: Zero chips			& Raise			& q2: Illegal state			& zeroChipsToIllegalState_raise()			\\ \hline
q3: Zero chips			& Check			& q2: Illegal state			& zeroChipsToIllegalState_check()			\\ \hline
q3: Zero chips			& Won round		& q0: Not in round			& zeroChipsToNotInRound_wonRound()		\\ \hline
q3: Zero chips			& Lost round		& q4: Out of game			& zeroChipsToNotInGame_lostRound()		\\ \hline

\end{tabular}
\end{table}

\avsnitt{Kodkritiksystem}


En presentation av de problem som hittats med hj�lp av verktyg f�r statisk analys och en diskussion av dem enligt anvisningarna. Det r�cker allts� inte med att bara lista problemen, ni m�ste f�rh�lla er till dem ocks�. T�nk ocks� p� att ni ska g�ra detta b�de p� koden som den s�g ut f�re granskningen och p� koden efter att ni r�ttat det som kommit fram under granskningen.

Vi k�rde FindBugs vid ett flertal tillf�llen under utvecklingen med olika resultat men den slutsatsen vi �nd� kunde dra av detta var att FindBugs var en av de minst anv�ndbara verktygen vi testat oss p�,
n�r den anv�ndes f�r f�rsta g�ngen precis innan den f�rsta granskningen s� hittade den inte ett enda fel, �ven n�r k�nsligheten drogs upp p� max och vi bad den rapportera allt den m�jligen kunde hitta.
Det fanns dock inga st�rre fel vid denna punkt men vi hittade �nd� r�tt mycket saker som vi beh�vde �ndra n�r vi k�rde v�r formella granskning som FindBugs ej hade hittat. M�nga av de felen vi hittade
hade dock med namngivning och liknande att g�ra vilket �r om�jligt f�r ett program att hitta. Vi k�rde ocks� den efter vi gjort korrigeringar efter granskningen men hittade d� heller inga fel. 
Vi k�rde d�remot FindBugs mot slutet av projektet ocks� och d� hittade den faktist n�gra intressanta fel som hj�lpte oss:
/*s�tt in fel som hittas nu i koden*/

\avsnitt{Statiska m�tt}

En presentation och diskussion kring ett antal l�mpliga statiska m�tt p� koden. Att vi inte specificerar exakt vilka m�tt som ska tas upp beror p� att olika verktyg har olika upps�ttningar, men vi f�rv�ntar oss fler och mer intressanta m�tt �n bara rena storleksm�tt som LOC, \#klasser, \#metoder, etc. �ven h�r �r det viktigt att f�rh�llas sig till m�tten, inte bara lista dem.

\textbf{Number of Classes}: 8 (exklusive testfall, interface och enums)
\begin{enumerate}
	\item cards: 5
	\item texasholdem: 3
\end{enumerate}

\textbf{Total Lines of Code}: 2014
\begin{enumerate}
	\item testfall: 1140 (56,6\%)
	\item texasholdem: 476  (23,6\%)
	\item cards: 398 (19,7\%)
\end{enumerate}

Det som kan vara intressant med resultaten f�r \emph{Number of Classes} och \emph{Total Lines of Code} �r det faktum att paketet texasholdem har f�rre klasser �n cards, men fler antal rader. Man b�r kanske fundera p� varf�r det har blivit s�, och fr�ga sig om det verkligen �r bra. Detta �r n�got som i sin tur st�r i korrelation till de metrics som f�ljer.

\textbf{Lack of Cohesion (mean)}:  0,403
\begin{enumerate}
	\item Game: 0,806
	\item Player: 0,795
	\item CardGame: 0,717
\end{enumerate}

\textbf{MacCabe Cyclomatic Complexity (mean)}: 1,632
\begin{enumerate}
	\item Pot:handOutChips() - 9
	\item TexasRules:hasCombination() - 9
	\item Game:placeBlinds() - 7
	\item Game:playerAction() - 7
\end{enumerate}

\textbf{Number of Parameters (mean)}: 0,667
\begin{enumerate}
	\item Pot:handOutChips() - 3
	\item Game:Game() - 3
\end{enumerate}


\textbf{Pizza metric index}: 1

\avsnitt{T�ckningsgrad}

En �versikt �ver vilken t�ckningsgrad era testfall uppn�tt. Denna kan antagligen tas rakt av fr�n verktyget ni anv�nt f�r att m�ta den. Om ni inte uppn�tt fullst�ndig t�ckning s� ska detta f�rklaras och motiveras.

\avsnitt{Profiler}

En kort presentation av hur ni g�tt tillv�ga f�r att testa koden med en profiler och vilka resultat ni fick fram.

\avsnitt{Byggscript}

Byggscriptets f�rsta (seri�sa) version, och den slutliga.

I den f�rsta versionen av byggscriptet vad ambitionen fr�mst att det skulle g�ra ungef�r samma sak som den standard-byggscripten som eclipse skapar f�r ett projekt. Med andra ord, kompilera alla klasser och dess beroenden. Det gick ganska bra fram till dess att testfallen inkluderades, d� det kr�vdes externa bibliotek (JUnit) f�r att kompilera dessa. S� sm�ningom fick vi �ven det att fungera.

 \lstset{
	language=XML,
	basicstyle=\ttfamily\scriptsize,
	keywordstyle=\color{blue}\ttfamily,
	stringstyle=\color{red}\ttfamily,
	showspaces=false,
	breaklines=true
 }

\begin{lstlisting}[float,caption={build.xml, f�rsta versionen},label=lst]
<?xml version="1.0"?>
<project name="texas" default="buildTest" basedir=".">
<property name="build.dir" location="bin" />
<property name="srcCards.dir" value="src/cards" />
<property name="srcTexasHoldem.dir" value="src/texasholdem" />
<property name="srcTest.dir" value="test" />

<property name="test.report.dir" location="testreports" />

<path id="junit.class.path">
	<pathelement location="${eclipse}/plugins/org.junit*/*junit.jar" />
	<pathelement location="/home/anton/dokument/eclipse/plugins/org.junit_4.11.0.v201303080030/junit.jar" />
	<pathelement location="${eclipse}/plugins/*org.hamcrest.*.jar" />
	<pathelement location="${build.dir}" />
</path>

<target name="buildCards" description="Build package cards">
	<mkdir dir="${build.dir}" />
	<javac includeantruntime="false" destdir="${build.dir}" source="1.7" target="1.7">
		<src path="${srcCards.dir}"/>
	</javac>
</target>

<target name="buildTexas" description="Build package texas" depends="buildCards">
	<javac includeantruntime="false" destdir="${build.dir}" source="1.7" target="1.7">
		<src path="${srcTexasHoldem.dir}"/>
	</javac>
</target>

<target name="buildTest" description="Build test cases" depends="buildTexas">
	<mkdir dir="${build.dir}/test" />
	<javac includeantruntime="false" destdir="${build.dir}/test" source="1.7" target="1.7">
		<classpath refid="junit.class.path" />
		<src path="${srcTest.dir}"/>
	</javac>
</target>
</project>
\end{lstlisting}

I den f�rdiga versionen s� k�rs �ven testfallen, och om desa g�r igenom skapas �ven en jar-fil f�r respektive paket.

\begin{lstlisting}[float,caption={build.xml, slutgiltliga versionen (1/2)},label=lst]
<?xml version="1.0"?>
<project name="texas" default="buildJarFiles" basedir=".">
<property name="build.dir" location="bin" />
<property name="cards.src.dir" location="src/cards" />
<property name="texasHoldem.src.dir" location="src/texasholdem" />
<property name="test.src.dir" location="test" />
<property name="test.report.dir" location="testreports" />
<property name="jar.dir" location="jar" />
<property name="eclipse" location="/home/anton/dokument/eclipse" />

<path id="junit.class.path">
	<fileset dir="${eclipse}/plugins" includes="org.junit*/*junit.jar" />
	<fileset dir="${eclipse}/plugins" includes="org.hamcrest*/*junit.jar" />
	<fileset dir="${eclipse}/plugins" includes="org.hamcrest.core*.jar" />
	<pathelement location="${build.dir}" />
</path>

<target name="clean" description="Delete previous jar and class files">
	<delete includeemptydirs="true" dir="jar" />
	<delete includeemptydirs="true" dir="bin" />
</target>

<target name="compileCards" description="Compile package cards" depends="clean">
	<mkdir dir="${build.dir}" />
	<javac includeantruntime="false" destdir="${build.dir}" source="1.7" target="1.7">
		<src path="${cards.src.dir}"/>
	</javac>
</target>

<target name="compileTexas" description="Compile package texasholdem" depends="compileCards">
	<javac includeantruntime="false" destdir="${build.dir}" source="1.7" target="1.7">
		<src path="${texasHoldem.src.dir}"/>
	</javac>
</target>
\end{lstlisting}

\begin{lstlisting}[float,caption={build.xml, slutgiltliga versionen (2/2)},label=lst]
<target name="compileTest" description="Compile test cases" depends="compileTexas, compileCards">
	<javac includeantruntime="false" destdir="${build.dir}" source="1.7" target="1.7">
		<classpath refid="junit.class.path" />
		<src path="${test.src.dir}"/>
	</javac>
</target>

<target name="runTestCases" description="Run test cases" depends="compileTest">
	<junit showoutput="true" printsummary="true" fork="yes" haltonfailure="yes">
		<classpath refid="junit.class.path" />
		<classpath>
			<pathelement location="${build.dir}"/>
		</classpath>
		<formatter type="plain" usefile="false" />
		<batchtest todir="${test.report.dir}">
			<fileset dir="${build.dir}">
				<include name="*Test.class" />
			</fileset>
		</batchtest>
	</junit>
</target>

<target name="buildJarFiles" description="Build .jar-files" depends="compileCards, compileTexas, compileTest, runTestCases">
	<mkdir dir="${jar.dir}" />
	<jar destfile="${jar.dir}/cards.jar" basedir="${build.dir}/cards" includes="*.class" />
	<jar destfile="${jar.dir}/texasholdem.jar" basedir="${build.dir}/texasholdem" includes="*.class" />
</target>

</project>
\end{lstlisting}



% Om ni vill f�r ni l�gga till ett avsnitt d�r ni tar upp �vrigt av relevans f�r bed�mningen av ert arbete. Detta ska i s� fall st� sist

% Nedanst�ende del ska bort i den inl�mnade versionen
\avsnitt{Mallanv�ndningen}

Detta avsnitt tar upp n�gra vanliga fr�gor/problem med mallandv�ndningen. Det ska naturligtvis plockas bort i den version ni l�mnar in.

\subsection*{Bilder}

Bilder ska naturligtvis tillf�ra n�got till texten, vara tydliga, ha en f�rklarande bildtext, och sitta p� ett vettigt st�lle i f�rh�llande till texten. De b�r ocks� refereras till i texten. T�nk p� att bilder i \LaTeX{} b�r placeras i floats, och att dessa kan flyttas runt automatiskt. Du m�ste allts� anta att bilden kan flyttas runt om du g�r n�gra �ndringar. Man b�r allts� alltid referera till figurer med hj�lp av labels: Figur \ref{fig:Exempelbild} p� sidan \pageref{fig:Exempelbild}.

\begin{figure}[htbp]

	\begin{center}
	\includegraphics[width=9cm]{money.pdf}
	\end{center}

	\caption{Exempelbild}
	\label{fig:Exempelbild}
\end{figure}

\subsection*{Rubrikniv�er}

Rubrikniv� tre och ner�t b�r man normalt undvika att anv�nda. Ibland g�r det inte, men ofta �r det ett tecken p� att man delar upp sitt dokument i f�r sm� bitar. P� denna uppgift finns det antagligen ingen anledning att anv�nda n�got annat �n den egendefinierade niv�n avsnitt som motsvarar section.


\subsection*{Kod}

Kod i dokumentet ska naturligtvis vara l�sbar och helst formaterad som i en utvecklingsmilj�. Ett tips �r att inte anv�nda sk�rmdumpar. De blir n�stan alltid ol�sliga, och det g�r inte att f� ut k�llkoden fr�n dem om man skulle vilja.

Mallen anv�nder sig av listingspaketet (\url{http://mirror.hmc.edu/ctan/macros/latex/contrib/listings/listings.pdf}) som kan formatera kod antingen direkt i latexkoden som i exempel \ref{lst}, eller fr�n en fil som i exempel \ref{lst:fil}. Man kan �ven plocka ut vissa rader ur en fil som i exempel \ref{lst:filrader}.

Inst�llningarna f�r programspr�k, radnumrering, etc.\ finns i direkt anslutning till d�r paketet importeras i latexmall.tex.

\begin{lstlisting}[float,caption={Kod som ligger i latexfilen},label=lst]
class Exempel{

	int number;

}
\end{lstlisting}

\lstinputlisting[float,caption={Kod som h�mtas fr�n fil},label=lst:fil]{Exempel.java}

\lstinputlisting[float,firstline=3,lastline=3,caption={En rad fr�n en fil},label=lst:filrader]{Exempel.java}






\end{document}
